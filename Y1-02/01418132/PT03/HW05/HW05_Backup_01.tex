\documentclass{article}

\usepackage{geometry}
\geometry{a4paper, portrait, margin = 1in}
\usepackage{multicol}
\usepackage{amsmath}
\usepackage{mathtools}
\usepackage[Latin, Thai]{ucharclasses}
\usepackage[thaifont = THSarabunNew]{thaispec}
\usepackage[edges]{forest}

\title{\textbf{Assignment V, Probability and Random Variables}}
\author{พศวัต ถิ่นกาญจน์วัฒนา \\
รหัสประจำตัวนิสิต 6410451199}
\date{}

\begin{document}

\maketitle

\flushleft
\renewcommand{\labelenumii}{\arabic{enumii})}

\begin{enumerate}

% 1
\item{หยิบไพ่ 2 ใบจากกองของไพ่ 8 ใบ ประกอบด้วย 4 queens และ 4 kings ถ้าไพ่อย่างน้อย 1 ใบ เป็น
queen \\
จงหาความน่าจะเป็นที่ไพ่ 2 ใบดังกล่าวเป็น queen ทั้งสองใบ}
\begin{align*}
|E| &= 3 \notag & |S| &= 7 \notag \\
\end{align*}
\begin{equation}
\begin{split}
p(E) &= \mathbf{\frac{3}{7}} \notag
\end{split}
\end{equation}

% 2
\item{จงหาความน่าจะเป็นในการเลือกตัวเลข 6 ตัว ที่ตรงกับรางวัลที่หนึ่งทุกตัว โดยลำดับของตัวเลขไม่มีความสำคัญ \\
กำหนดให้จำนวนเลขที่เลือกมีค่าตั้งแต่ 1 จนถึงค่าต่อไปนี้}
	\begin{enumerate}
	
	% 2.1
	\item{40}
	\begin{align*}
	|E| &= 1 \notag & |S| &= ~^{40}C_6 \notag \\
	&& &= \frac{40!}{(40 - 6)! \cdot 6!} \\
	&& &= 3,838,380
	\end{align*}
	\begin{equation}
	\begin{split}
	p(E) &= \frac{|E|}{|S|} \notag \\
	&= \mathbf{\frac{1}{3,838,380}}
	\end{split}
	\end{equation}
	
	% 2.2
	\item{48}
	\begin{align*}
	|E| &= 1 & |S| &= ~^{48}C_6 \notag \\
	&& &= \frac{48!}{(48 - 6)! \cdot 6!} \\
	&& &= 12,271,512
	\end{align*}
	\begin{equation}
	\begin{split}
	p(E) &= \frac{|E|}{|S|} \notag \\
	&= \mathbf{\frac{1}{12,271,512}}
	\end{split}
	\end{equation}
	
	% 2.3
	\item{56}
	\begin{align*}
	|E| &= 1 & |S| &= ~^{56}C_6 \notag \\
	&& &= \frac{56!}{(56 - 6)! \cdot 6!} \\
	&& &= 32,468,436 \\
	\end{align*}
	\begin{equation}
	\begin{split}
	p(E) &= \frac{|E|}{|S|} \notag \\
	&= \mathbf{\frac{1}{32,468,436}}
	\end{split}
	\end{equation}
	
	% 2.4
	\item{64}
	\begin{align*}
	|E| &= 1 & |S| &= ~^{64}C_6 \notag \\
	&& &= \frac{64!}{(64 - 6)! \cdot 6!} \\
	&& &= 74,974,368
	\end{align*}
	\begin{equation}
	\begin{split}
	p(E) &= \frac{|E|}{|S|} \notag \\
	&= \mathbf{\frac{1}{74,974,368}}
	\end{split}
	\end{equation}
	
	\end{enumerate}
	
% 3
\item{จงหาความน่าจะเป็นในการเลือกไพ่ 5 ใบ ที่ประกอบด้วย 2 ข้าวหลามตัด 3 โพดำ 6 โพแดง 10 ดอกจิก \\
และ King โพแดง}
\begin{align*}
|E| &= 51^4 \times 4 & |S| &= 52^5 \notag \\
&= 27,060,804 & &= 380,204,032
\end{align*}
\begin{equation}
\begin{split}
p(E) &= \frac{|E|}{|S|} \notag \\
&= \frac{27,060,804}{380,204,032} \\
&\approx \mathbf{0.07117442668256606}
\end{split}
\end{equation}

% 4
\item{จงหาความน่าจะเป็นในการทอยลูกเต๋าที่สมดุล 1 ลูก จำนวน 6 ครั้ง โดยไม่ปรากฏหน้าที่เป็นเลขคู่เลย}
\begin{align*}
|E| &= 3^6 & |S| &= 6^6 \notag \\
&= 729 & &= 46,656
\end{align*}
\begin{equation}
\begin{split}
p(E) &= \frac{|E|}{|S|} \notag \\
&= \frac{729}{46,656} \\
&= \mathbf{0.015625}
\end{split}
\end{equation}

\pagebreak

% 5
\item{จงหาความน่าจะเป็นแบบมีเงื่อนไขที่บิตสตริงความยาว 4 มีบิต 0 ติดกันอย่างน้อย 2 ตัว \\
โดยกำหนดให้บิตแรกมีค่าเป็น 1}
\begin{align*}
|E| &= 2^2 & |S| &= 2^4 \notag \\
&= 4 & &= 16
\end{align*}
\begin{equation}
\begin{split}
p(E) &= \frac{|E|}{|S|} \notag \\
&= \frac{4}{16} \\
&= \frac{1}{4} \\
&= \mathbf{0.25}
\end{split}
\end{equation}

% 6
\item{ครอบครัวหนึ่งมีลูก 5 คน จงหาความน่าจะเป็นที่ลูกคนแรกเป็นผู้ชาย หรือสองคนสุดท้ายเป็นผู้หญิง เมื่อ}
	\begin{enumerate}
	
	% 6.1
	\item{ผู้ชายหรือผู้หญิงมีความน่าจะเป็นเท่ากัน}
	\begin{multicols}{2}
	\begin{align*}
	|E_1| &= 2^4 & |S_1| &= 2^5 \notag \\
	&= 4 & &= 32
	\end{align*}
	\begin{equation}
	\begin{split}
	p(E_1) &= \frac{16}{32} \notag \\
	&= 0.5
	\end{split}
	\end{equation}

	\columnbreak

	\begin{align*}
	|E_2| &= 2^3 & |S_2| &= 2^5 \notag \\
	&= 8 & &= 32
	\end{align*}
	\begin{equation}
	\begin{split}
	p(E_2) &= \frac{8}{32} \notag \\
	&= 0.25
	\end{split}
	\end{equation}

	\end{multicols}

	\begin{equation}
	\begin{split}
	p(E) &= p(E_1) + p(E_2) \notag \\
	&= 0.5 + 0.25 \\
	&= \mathbf{0.75}
	\end{split}
	\end{equation}
	
	% 6.2
	\item{ความน่าจะเป็นของผู้ชายเป็น 0.51}
	\begin{multicols}{2}
	\begin{align*}
	|E_1| &= 2^4 & |S_1| &= 2^5 \notag \\
	&= 4 & &= 32
	\end{align*}
	\begin{equation}
	\begin{split}
	p(E_1) &= \frac{16}{32} \notag \\
	&= 0.5
	\end{split}
	\end{equation}

	\columnbreak

	\begin{align*}
	|E_2| &= 2^3 & |S_2| &= 2^5 \notag \\
	&= 8 & &= 32
	\end{align*}
	\begin{equation}
	\begin{split}
	p(E_2) &= \frac{8}{32} \notag \\
	&= 0.25
	\end{split}
	\end{equation}
	
	\end{multicols}

	\begin{equation}
	\begin{split}
	p(E) &= 0.51(p(E_1)) + p(E_2) \notag \\
	&= 0.51(0.5) + 0.25 \\
	&= 0.255 + 0.25 \\
	&= \mathbf{0.505}
	\end{split}
	\end{equation}
	
	\end{enumerate}

% 7
\item{สมมติให้ 2\% ของคนที่ไม่ใช้ฝิ่นมีผลทดสอบการใช้ฝิ่นเป็น positive (หรือเรียกอีกอย่างหนึ่งว่า 2\% เป็น \\
false positive) และ 5\% ของคนที่ใช้ฝิ่นมีผลทดสอบการใช้ฝิ่นเป็น negative (หรือเรียกอีกอย่างหนึ่งว่า
5\% เป็น \\
false negative) และมีคน 1\% ที่เป็นผู้ใช้ฝิ่น จงหา}
	\begin{enumerate}
	
	% 7.1
	\item{ความน่าจะเป็นที่คนที่ทดสอบการใช้ฝิ่นได้ผล negative ไม่ได้ใช้ฝิ่น}
	\begin{equation}
	\begin{split}
	\notag \\
	\end{split}
	\end{equation}
	
	% 7.2
	\item{ความน่าจะเป็นที่คนที่ทดสอบการใช้ฝิ่นได้ผล positive เป็นผู้ใช้ฝิ่น}
	\begin{equation}
	\begin{split}
	\notag \\
	\end{split}
	\end{equation}
	
	\end{enumerate}
% 8
\item{สมมติให้คนไข้ 8\% เป็นผู้ติดเชื้อไวรัสสายพันธุ์ใหม่กว่า นอกจากนี้พบว่า 98\% ของคนไข้ที่เป็นผู้ติดเชื้อไวรัส \\
สายพันธุ์ใหม่กว่ามีผลการทดสอบเป็น positive และ 3\% ของคนไข้ที่ไม่เป็นผู้ติดเชื้อไวรัสสายพันธุ์ใหม่กว่า \\
มีผลการทดสอบเป็น positive จงหา}
	\begin{enumerate}
	
	% 8.1
	\item{ความน่าจะเป็นที่คนไข้ที่มีผลการทดสอบ positive เป็นผู้ติดเชื้อ}
	\begin{equation}
	\begin{split}
	\notag \\
	\end{split}
	\end{equation}
	
	% 8.2
	\item{ความน่าจะเป็นที่คนไข้ที่มีผลการทดสอบ positive ไม่เป็นผู้ติดเชื้อ}
	\begin{equation}
	\begin{split}
	\notag \\
	\end{split}
	\end{equation}
	
	% 8.3
	\item{ความน่าจะเป็นที่คนไข้ที่มีผลการทดสอบ negative เป็นผู้ติดเชื้อ}
	\begin{equation}
	\begin{split}
	\notag \\
	\end{split}
	\end{equation}
	
	% 8.4
	\item{ความน่าจะเป็นที่คนไข้ที่มีผลการทดสอบ negative ไม่เป็นผู้ติดเชื้อ}
	\begin{equation}
	\begin{split}
	\notag \\
	\end{split}
	\end{equation}
	
	\end{enumerate}

% 9
\item{จงหา expected value ของผลรวมของการโยนลูกเต๋า 2 ลูก โดยลูกเต๋าแต่ละลูกมี biased ที่จะออกเลข 3 \\
เป็นสองเท่าของเลขอื่น ๆ}
\begin{equation}
\begin{split}
\notag \\
\end{split}
\end{equation}

% 10
\item{ในการสอบวิชา 01418000 ข้อสอบประกอบด้วยคำถามแบบถูก/ผิด จำนวน 50 ข้อ ข้อละ 2 คะแนน \\
และคำถามแบบเลือกตอบ 4 ตัวเลือก จำนวน 25 ข้อ ข้อละ 4 คะแนน ความน่าจะเป็นที่แป๋งตอบคำถาม \\
แบบ
ถูก/ผิดได้ถูกต้องคิดเป็น 0.9 และความน่าจะเป็นที่จะตอบคำถามแบบเลือกตอบได้ถูกต้องคิดเป็น 0.8 \\
จงหา expected value ของคะแนนที่แป๋งจะได้}
\begin{equation}
\begin{split}
\notag \\
\end{split}
\end{equation}

% 11
\item{ป๋องใส่ลูกบอล m ลูกลงในถัง n ใบแบบสุ่มและกระจายไปในถังทุก ๆ ใบเท่า ๆ กัน จงหา}
	\begin{enumerate}
	
	% 11.1
	\item{ความน่าจะเป็นที่ถังใบแรกไม่มีลูกบอล}
	\begin{equation}
	\begin{split}
	\notag \\
	\end{split}
	\end{equation}
	
	% 11.2
	\item{Expected value ของจำนวนถังที่ไม่มีลูกบอล}
	\begin{equation}
	\begin{split}
	\notag \\
	\end{split}
	\end{equation}
	
	\end{enumerate}

% 12
\item{จงหาค่า variance ของจำนวน success ในการทำ Bernoulli trials จำนวน n ครั้งที่แต่ละครั้งมีความน่าจะเป็น \\
ของ success เป็น p และความน่าจะเป็นของ failure เป็น q}
\begin{equation}
\begin{split}
\notag \\
\end{split}
\end{equation}

\end{enumerate}

\end{document}