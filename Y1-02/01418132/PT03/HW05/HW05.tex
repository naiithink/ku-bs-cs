\documentclass{article}

\usepackage{geometry}
\geometry{a4paper, portrait, margin = 1in}
\usepackage{multicol}
\usepackage{amsmath}
\usepackage{amssymb}
\usepackage{mathtools}
\usepackage{venndiagram}
\usepackage[Latin, Thai]{ucharclasses}
\usepackage[thaifont = THSarabunNew]{thaispec}
\usepackage[edges]{forest}

\setlength{\columnseprule}{0.4pt}

\title{\textbf{Assignment V, Probability and Random Variables}}
\author{พศวัต ถิ่นกาญจน์วัฒนา \\
รหัสประจำตัวนิสิต 6410451199}
\date{}

\begin{document}

\maketitle

\flushleft
\renewcommand{\labelenumii}{\arabic{enumii})}

\begin{enumerate}

% 1
\item{หยิบไพ่ 2 ใบจากกองของไพ่ 8 ใบ ประกอบด้วย 4 queens และ 4 kings ถ้าไพ่อย่างน้อย 1 ใบ เป็น
queen \\
จงหาความน่าจะเป็นที่ไพ่ 2 ใบดังกล่าวเป็น queen ทั้งสองใบ}
\begin{align*}
|E|	&= 3		&		|S|	&= 7		&		p(E)	&= \frac{|E|}{|S|}			\notag \\
	&			&			&			&				&= \mathbf{\frac{3}{7}}
\end{align*}

% 2
\item{จงหาความน่าจะเป็นในการเลือกตัวเลข 6 ตัว ที่ตรงกับรางวัลที่หนึ่งทุกตัว โดยลำดับของตัวเลขไม่มีความสำคัญ \\
กำหนดให้จำนวนเลขที่เลือกมีค่าตั้งแต่ 1 จนถึงค่าต่อไปนี้}
	\begin{enumerate}
	
	% 2.1
	\item{40}
	\begin{align*}
	|E|	&= 1		&		|S|	&= ~^{40}C_6							&		p(E)	&= \frac{|E|}{|S|}				\notag \\
		&			&			&= \frac{40!}{(40 - 6)! \cdot 6!}		&				&= \mathbf{\frac{1}{3,838,380}}	\\
		&			&			&= 3,838,380
	\end{align*}
	
	% 2.2
	\item{48}
	\begin{align*}
	|E|	&= 1		&		|S|	&= ~^{48}C_6							&		p(E)	&= \frac{|E|}{|S|}					\notag \\
		&			&			&= \frac{48!}{(48 - 6)! \cdot 6!}		&				&= \mathbf{\frac{1}{12,271,512}}	\\
		&			&			&= 12,271,512
	\end{align*}
	
	% 2.3
	\item{56}
	\begin{align*}
	|E|	&= 1		&		|S|	&= ~^{56}C_6							&		p(E)	&= \frac{|E|}{|S|}					\notag \\
		&			&			&= \frac{56!}{(56 - 6)! \cdot 6!}		&				&= \mathbf{\frac{1}{32,468,436}}	\\
		&			&			&= 32,468,436
	\end{align*}
	
	% 2.4
	\item{64}
	\begin{align*}
	|E|	&= 1		&		|S|	&= ~^{64}C_6							&		p(E)	&= \frac{|E|}{|S|}					\notag \\
		&			&			&= \frac{64!}{(64 - 6)! \cdot 6!}		&				&= \mathbf{\frac{1}{74,974,368}}	\\
		&			&			&= 74,974,368
	\end{align*}
	
	\end{enumerate}
	
% 3
\item{จงหาความน่าจะเป็นในการเลือกไพ่ 5 ใบ ที่ประกอบด้วย 2 ข้าวหลามตัด 3 โพดำ 6 โพแดง 10 ดอกจิก \\
และ King โพแดง}
\begin{align*}
|E|	&= 51^4 \times 4		&		|S|	&= 52^5				&		p(E)	&= \frac{|E|}{|S|}					\notag \\
	&= 27,060,804			&			&= 380,204,032		&	 			&= \frac{27,060,804}{380,204,032}	\\
	&						&			&					&				&\approx \mathbf{0.0711744266}
\end{align*}

% 4
\item{จงหาความน่าจะเป็นในการทอยลูกเต๋าที่สมดุล 1 ลูก จำนวน 6 ครั้ง โดยไม่ปรากฏหน้าที่เป็นเลขคู่เลย}
\begin{align*}
|E|	&= 3^6		&		|S|	&= 6^6		&		p(E)	&= \frac{|E|}{|S|}		\notag \\
	&= 729		&			&= 46,656	&				&= \frac{729}{46,656}	\\
	&			&			&			&				&= \mathbf{0.015625}
\end{align*}

% 5
\item{จงหาความน่าจะเป็นแบบมีเงื่อนไขที่บิตสตริงความยาว 4 มีบิต 0 ติดกันอย่างน้อย 2 ตัว \\
โดยกำหนดให้บิตแรกมีค่าเป็น 1}
\begin{align*}
|E| &= 2^2		&		|S|	&= 2^4		&		p(E)	&= \frac{|E|}{|S|}		\notag \\
	&= 4 		&	 		&= 16 		&	 			&= \frac{4}{16}			\\
	&			&			&			&				&= \frac{1}{4}			\\
	&			&			&			&				&= \mathbf{0.25}
\end{align*}

% 6
\item{ครอบครัวหนึ่งมีลูก 5 คน จงหาความน่าจะเป็นที่ลูกคนแรกเป็นผู้ชาย หรือสองคนสุดท้ายเป็นผู้หญิง เมื่อ}
	\begin{enumerate}
	
	% 6.1
	\item{ผู้ชายหรือผู้หญิงมีความน่าจะเป็นเท่ากัน}
	\begin{align*}
	Define	&&		\text{ความน่าจะเป็นของผู้ชาย}	&::=	p(E_1)			&&					\\
	Define	&&		\text{ความน่าจะเป็นของผู้หญิง}	&::=	p(E_2)			&&					\\
	Let		&&		p(E_1)						&= p(E_2)				&& \tag{1}			\\
			&&		1							&= p(E_1) + p(E_2)		&&					\\
			&&		1							&= 2 \cdot p(E_1)		&& ; \tag{1} 		\\
			&&		p(E_1)						&= \frac{1}{2}			&& \\
			&&		p(E_1)						&= p(E_2) = 0.5			&& \tag{2}
	\end{align*}

\pagebreak

	\begin{multicols}{2}
	\noindent
		\begin{align*}
		|E_1| &= 2^4 & |S_1| &= 2^5 \notag \\
		&= 4 & &= 32
		\end{align*}
		\begin{equation}
		\begin{split}
		p(E_1) &= \frac{16}{32} \notag \\
		&= 0.5
		\end{split}
		\end{equation}
	\columnbreak
		\begin{align*}
		|E_2| &= 2^3 & |S_2| &= 2^5 \notag \\
		&= 8 & &= 32
		\end{align*}
		\begin{equation}
		\begin{split}
		p(E_2) &= \frac{8}{32} \notag \\
		&= 0.25
		\end{split}
		\end{equation}
	\end{multicols}
	\noindent
	\begin{equation}
	\begin{split}
	p(E) &= p(E_1) + p(E_2) \notag \\
	&= 0.5 + 0.25 \\
	&= \mathbf{0.75}
	\end{split}
	\end{equation}

	% 6.2
	\item{ความน่าจะเป็นของผู้ชายเป็น 0.51}
	\begin{align*}
	Define	&&		\text{ความน่าจะเป็นของผู้ชาย}	&::=	p(E_1)				&&					\\
	Define	&&		\text{ความน่าจะเป็นของผู้หญิง}	&::=	p(E_2)				&&					\\
	Let		&&		p(E_1)						&=		0.51				&& \tag{1}			\\
			&&		1							&=		p(E_1) + p(E_2)		&&					\\
			&&		1							&=		0.51 + p(E_2)		&& ; \tag{1}		\\
			&&		p(E_2)						&= 		1 - 0.51			&&					\\
			&&									&= 		0.49
	\end{align*}
	\begin{multicols}{2}
	\noindent
		\begin{align*}
		|E_1| &= 2^4 & |S_1| &= 2^5 \notag \\
		&= 4 & &= 32
		\end{align*}
		\begin{equation}
		\begin{split}
		p(E_1) &= \frac{16}{32} \notag \\
		&= 0.5
		\end{split}
		\end{equation}
	\columnbreak
		\begin{align*}
		|E_2| &= 2^3 & |S_2| &= 2^5 \notag \\
		&= 8 & &= 32
		\end{align*}
		\begin{equation}
		\begin{split}
		p(E_2) &= \frac{8}{32} \notag \\
		&= 0.25
		\end{split}
		\end{equation}
	\end{multicols}
	\noindent
	\begin{equation}
	\begin{split}
	p(E) &= 0.51(p(E_1)) + p(E_2) \notag \\
	&= 0.51(0.5) + 0.25 \\
	&= 0.255 + 0.25 \\
	&= \mathbf{0.505}
	\end{split}
	\end{equation}
	
	\end{enumerate}

\pagebreak

% 7
\item{สมมติให้ 2\% ของคนที่ไม่ใช้ฝิ่นมีผลทดสอบการใช้ฝิ่นเป็น positive (หรือเรียกอีกอย่างหนึ่งว่า 2\% เป็น \\
false positive) และ 5\% ของคนที่ใช้ฝิ่นมีผลทดสอบการใช้ฝิ่นเป็น negative (หรือเรียกอีกอย่างหนึ่งว่า
5\% เป็น \\
false negative) และมีคน 1\% ที่เป็นผู้ใช้ฝิ่น จงหา}
\begin{align*}
\text{กำหนดให้} &&& \mathbb{U}~ \text{เป็น}~ universe~ \text{และ}~ |\mathbb{U}| & &= 100 \\
\text{กำหนดให้} &&& \text{เซตของคนที่ใช้ฝิ่น} & &::= A \\
\text{กำหนดให้} &&& \text{เซตของคนที่ไม่ใช้ฝิ่น} & &::= B \\
\text{กำหนดให้} &&& \text{เซตของคนที่ได้ผลการทดสอบเป็น positive} & &::= C \\
\text{กำหนดให้} &&& \text{เซตของคนที่ได้ผลการทดสอบเป็น negative} & &::= D \\
\text{โจทย์กำหนด} &&& 2\%~ \text{เป็น}~ false~positive: && |B \cap C| = 2 \\
\text{โจทย์กำหนด} &&& 5\%~ \text{เป็น}~ false~negative: && |A \cap D| = 5 \\
\text{โจทย์กำหนด} &&& 1\%~ \text{เป็นคนที่ใช้ฝิ่น}: && |A| = 1
\end{align*}
	\begin{enumerate}
	
	% 7.1
	\item{ความน่าจะเป็นที่คนที่ทดสอบการใช้ฝิ่นได้ผล negative ไม่ได้ใช้ฝิ่น}
	\begin{equation}
	\begin{split}
	true~negative &= B \cap D \notag \\
	|A| &= 1 \\
	|A'| &= |\mathbb{U} - A| \\
	&= 100 - 1 \\
	&= 99
	\end{split}
	\end{equation}
	
	% 7.2
	\item{ความน่าจะเป็นที่คนที่ทดสอบการใช้ฝิ่นได้ผล positive เป็นผู้ใช้ฝิ่น}
	\begingroup
	\begin{equation}
	\begin{split}
	\notag \\
	\end{split}
	\end{equation}
	\endgroup
	
	\end{enumerate}
% 8
\item{สมมติให้คนไข้ 8\% เป็นผู้ติดเชื้อไวรัสสายพันธุ์ใหม่กว่า นอกจากนี้พบว่า 98\% ของคนไข้ที่เป็นผู้ติดเชื้อไวรัส \\
สายพันธุ์ใหม่กว่ามีผลการทดสอบเป็น positive และ 3\% ของคนไข้ที่ไม่เป็นผู้ติดเชื้อไวรัสสายพันธุ์ใหม่กว่า \\
มีผลการทดสอบเป็น positive จงหา}
	\begin{enumerate}
	
	% 8.1
	\item{ความน่าจะเป็นที่คนไข้ที่มีผลการทดสอบ positive เป็นผู้ติดเชื้อ}
	\begin{equation}
	\begin{split}
	\notag \\
	\end{split}
	\end{equation}
	
	% 8.2
	\item{ความน่าจะเป็นที่คนไข้ที่มีผลการทดสอบ positive ไม่เป็นผู้ติดเชื้อ}
	\begin{equation}
	\begin{split}
	\notag \\
	\end{split}
	\end{equation}
	
	% 8.3
	\item{ความน่าจะเป็นที่คนไข้ที่มีผลการทดสอบ negative เป็นผู้ติดเชื้อ}
	\begin{equation}
	\begin{split}
	\notag \\
	\end{split}
	\end{equation}
	
	% 8.4
	\item{ความน่าจะเป็นที่คนไข้ที่มีผลการทดสอบ negative ไม่เป็นผู้ติดเชื้อ}
	\begin{equation}
	\begin{split}
	\notag \\
	\end{split}
	\end{equation}
	
	\end{enumerate}

% 9
\item{จงหา expected value ของผลรวมของการโยนลูกเต๋า 2 ลูก โดยลูกเต๋าแต่ละลูกมี biased ที่จะออกเลข 3 \\
เป็นสองเท่าของเลขอื่น ๆ}
\begin{equation}
\begin{split}
\notag \\
\end{split}
\end{equation}

% 10
\item{ในการสอบวิชา 01418000 ข้อสอบประกอบด้วยคำถามแบบถูก/ผิด จำนวน 50 ข้อ ข้อละ 2 คะแนน \\
และคำถามแบบเลือกตอบ 4 ตัวเลือก จำนวน 25 ข้อ ข้อละ 4 คะแนน ความน่าจะเป็นที่แป๋งตอบคำถาม \\
แบบ
ถูก/ผิดได้ถูกต้องคิดเป็น 0.9 และความน่าจะเป็นที่จะตอบคำถามแบบเลือกตอบได้ถูกต้องคิดเป็น 0.8 \\
จงหา expected value ของคะแนนที่แป๋งจะได้}
\begin{equation}
\begin{split}
\notag \\
\end{split}
\end{equation}

% 11
\item{ป๋องใส่ลูกบอล m ลูกลงในถัง n ใบแบบสุ่มและกระจายไปในถังทุก ๆ ใบเท่า ๆ กัน จงหา}
	\begin{enumerate}
	
	% 11.1
	\item{ความน่าจะเป็นที่ถังใบแรกไม่มีลูกบอล}
	\begin{equation}
	\begin{split}
	\notag \\
	\end{split}
	\end{equation}
	
	% 11.2
	\item{Expected value ของจำนวนถังที่ไม่มีลูกบอล}
	\begin{equation}
	\begin{split}
	\notag \\
	\end{split}
	\end{equation}
	
	\end{enumerate}

% 12
\item{จงหาค่า variance ของจำนวน success ในการทำ Bernoulli trials จำนวน n ครั้งที่แต่ละครั้งมีความน่าจะเป็น \\
ของ success เป็น p และความน่าจะเป็นของ failure เป็น q}
\begin{equation}
\begin{split}
\notag \\
\end{split}
\end{equation}

\end{enumerate}

\end{document}