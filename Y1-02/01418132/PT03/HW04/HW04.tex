\documentclass{article}

\usepackage{amsmath}
\usepackage{geometry}
\geometry{a4paper, portrait, margin = 1in}
\usepackage[thaifont = THSarabunNew]{thaispec}
\usepackage[edges]{forest}

\title{\textbf{Assignment IV, Basic Counting}}
\author{พศวัต ถิ่นกาญจน์วัฒนา \\
รหัสประจำตัวนิสิต 6410451199}
\date{}

\begin{document}

\maketitle

\flushleft
\renewcommand{\labelenumii}{\arabic{enumii})}

\begin{enumerate}
	% 1
	\item จงหาจำนวนวิธีทั้งหมดในการชนะ 3 เกมจากทั้งหมด 5 เกม ระหว่างผู้เล่น 2 คน คือ $T_1$ และ $T_2$\newline
	และอยากทราบว่าผู้เล่นทั้ง 2 คน จะสามารถชนะเกมได้ทีมละกี่เกม (อาจใช้ Tree diagram)
	
	\begin{forest}
	for tree = {
		tier/.option = level,
		anchor = mid,
		grow = east,
		s sep = 1.5mm,
		l sep = 10mm,
		inner sep = 0.7mm
	}
		[$T_1$ / $T_2$, rectangle, draw = black, inner sep = 1mm
			[$L$
				[$L$
					[$L$
						[$L$
							[$L$ [$LLLLL$, name = i1, tier = word]]
							[$W$ [$LLLLW$, name = i2, tier = word]]
						]
						[$W$
							[$L$ [$LLLWL$, tier = word]]
							[$W$ [$LLLWW$, tier = word]]
						]
					]
					[$W$
						[$L$
							[$L$ [$LLWLL$, tier = word]]
							[$W$ [$LLWLW$, tier = word]]
						]
						[$W$
							[$L$ [$LLWWL$, tier = word]]
							[$W$ [$LLWWW$, draw, red, tier = word]]
						]
					]
				]
				[$W$
					[$L$
						[$L$
							[$L$ [$LWLLL$, tier = word]]
							[$W$ [$LWLLW$, tier = word]]
						]
						[$W$
							[$L$ [$LWLWL$, tier = word]]
							[$W$ [$LWLWW$, draw, red, tier = word]]
						]
					]
					[$W$
						[$L$
							[$L$ [$LWWLL$, tier = word]]
							[$W$ [$LWWLW$, draw, red, tier = word]]
						]
						[$W$
							[$L$ [$LWWWL$, draw, red, tier = word]]
							[$W$ [$LWWWW$, draw, red, tier = word]]
						]
					]
				]
			]
			[$W$
				[$L$
					[$L$
						[$L$
							[$L$ [$WLLLL$, tier = word]]
							[$W$ [$WLLLW$, tier = word]]
						]
						[$W$
							[$L$ [$WLLWL$, tier = word]]
							[$W$ [$WLLWW$, draw, red, tier = word]]
						]
					]
					[$W$
						[$L$
							[$L$ [$WLWLL$, tier = word]]
							[$W$ [$WLWLW$, draw, red, tier = word]]
						]
						[$W$
							[$L$ [$WLWWL$, draw, red, tier = word]]
							[$W$ [$WLWWW$, draw, red, tier = word]]
						]
					]
				]
				[$W$
					[$L$
						[$L$
							[$L$ [$WWLLL$, tier = word]]
							[$W$ [$WWLLW$, draw, red, tier = word]]
						]
						[$W$
							[$L$ [$WWLWL$, draw, red, tier = word]]
							[$W$ [$WWLWW$, draw, red, tier = word]]
						]
					]
					[$W$
						[$L$
							[$L$ [$WWWLL$, draw, red, tier = word]]
							[$W$ [$WWWLW$, draw, red, tier = word]]
						]
						[$W$
							[$L$ [$WWWWL$, name = o2, draw, red, tier = word]]
							[$W$ [$WWWWW$, name = o1, draw, red, tier = word]]
						]
					]
				]
			]
		]
		\node (g1) at (2, 9.5) {เกมที่ 1};
		\node (g2) at (3.5, 9.5) {เกมที่ 2};
		\node (g3) at (5, 9.5) {เกมที่ 3};
		\node (g4) at (6.5, 9.5) {เกมที่ 4};
		\node (g5) at (8, 9.5) {เกมที่ 5};
		% \draw [->] (o1) to [out = east, in = east] (i1);
		% \draw [->] (o2) to [out = east, in = east] (i2);
	\end{forest}
	
	% 2
	\item จงหาจำนวนวิธีทั้งหมดในการจัดที่นั่งให้คน 4 คนในโต๊ะกลม โดยการจัดที่นั่งที่คนด้านซ้ายและขวาเป็นคนเดิม\newline
	ถือว่าเป็นการจัดที่นั่งซ้ำ
        	\begin{equation}
        	\begin{split}
	circular~arrangement~of~n-object ::&= (n - 1)! \notag\\
        &= (4 - 1)! \\
        &= \mathbf{6}
        	\end{split}
        	\end{equation}
	
	% 3
	\item จงหาจำนวนสมาชิกของ $|A \cup B|$ เมื่อ $|A| = 12$ และ $|B| = 18$ เมื่อแต่ละกรณีต่อไปนี้เป็นจริง
	\begin{enumerate}
		\item $|A \cap B| = \phi$
		\begin{equation}
		\begin{split}
		|A \cup B| &= |A| + |B| - |A \cap B| \notag\\
		&= 12 + 18 - 0 \\
		&=  \mathbf{30}
		\end{split}
		\end{equation}
		
		\item $|A \cap B| = 1$
		\begin{equation}
		\begin{split}
		|A \cup B| &= |A| + |B| - |A \cap B| \notag\\
		&= 12 + 18 - 1 \\
		&=  \mathbf{29}
		\end{split}
		\end{equation}
		
		\item $|A \cap B| = 6$
		\begin{equation}
		\begin{split}
		|A \cup B| &= |A| + |B| - |A \cap B| \notag\\
		&= 12 + 18 - 6 \\
		&=  \mathbf{24}
		\end{split}
		\end{equation}
		
		\item $A \subseteq B$
		\begin{equation}
		\begin{split}
		|A \cup B| &= |A| + |B| - |A \cap B| \notag\\
		&= 12 + 18 - 12 \\
		&=  \mathbf{18}
		\end{split}
		\end{equation}
		
	\end{enumerate}
	
	% 4
	\item อักขระภาษาอังกฤษประกอบด้วยพยัญชนะ 21 ตัว สระ 5 ตัว จงหาสตริงของอักขระตัวเล็ก (lowercase)\newline
	ความยาว 6 ที่ประกอบด้วย
	\begin{enumerate}
		\item มีสระ 1 ตัว
		\begin{equation}
		\begin{split}
		_{21}C_{5} ~\times~ _{5}C_{1} &= \frac{21!}{16! \times 5!} \times \frac{5!}{4! \times 1!} \notag\\
		&= 20,349 \times 5 \\
		&=  \mathbf{101,745~strings}
		\end{split}
		\end{equation}
		
		\item มีสระ 2 ตัว โดยสามารถเลือกสระซ้ำได้
		\begin{equation}
		\begin{split}
		_{21}C_{4} ~\times~ _{5}C_{2} &= \frac{21!}{17! \times 4!} \times \frac{5!}{3! \times 2!} \notag\\
		&= 5,985 \times 10 \\
		&=  \mathbf{59,850~strings}
		\end{split}
		\end{equation}
		
		\item มีสระอย่างน้อย 1 ตัว
		\begin{equation}
		\begin{split}
		 (_5C_1 ~\times~ _{21}C_5) + (_5C_2 ~\times~ _{21}C_4) + (_5C_3 ~\times~ _{21}C_3) + (_5C_4 ~\times~ _{21}C_2) + (_5C_5 ~\times~ _{21}C_1) + (_5C_5 ~\times~ 5) \notag \\
		\end{split}
		\end{equation}
		
		\item มีสระอย่างน้อย 2 ตัว
		\begin{equation}
		\begin{split}
		 (_5C_2 ~\times~ _{21}C_4) + (_5C_3 ~\times~ _{21}C_3) + (_5C_4 ~\times~ _{21}C_2) + (_5C_5 ~\times~ _{21}C_1) + (_5C_5 ~\times~ 5) \notag \\
		\end{split}
		\end{equation}
	\end{enumerate}
	
	% 5
	\item การจับสลากของเลข 1 ถึง 100 เพื่อมอบรางวัลจำนวน 4 รางวัล ประกอบด้วยรางวัลที่ 1, 2, 3 และรางวัลพิเศษ\newline
	จงหาจำนวนวิธีในการมอบรางวัลทั้งสี่ภายใต้เงื่อนไข ดังนี้
	\begin{enumerate}
		\item ไม่มีกติกาเพิ่มเติม
		\begin{equation}
		\begin{split}
		_{100}C_{4} &= \frac{100!}{(100 - 4)!} \notag\\
		&= \frac{100!}{96!} \\
		&= 97 \times 98 \times 99 \times 100 \\
		&=  \mathbf{94,109,400~ways}
		\end{split}
		\end{equation}
		
		\item ผู้ที่ถือสลากหมายเลข 47 ได้รับรางวัลพิเศษ
		\begin{equation}
		\begin{split}
		_{99}C_{3} &= \frac{99!}{(99 - 3)!} \notag \\
		&= \frac{99!}{96!} \\
		&= 97 \times 98 \times 99 \\
		&=  \mathbf{941,094~ways}
		\end{split}
		\end{equation}
		
		\item ผู้ที่ถือสลากหมายเลข 47 ได้รับรางวัลใดรางวัลหนึ่งในสี่รางวัล
		\begin{equation}
		\begin{split}
		_{99}C_{3} &= \frac{99!}{(99 - 3)!} \times 4 \notag \\
		&= \frac{99!}{96!} \times 4 \\
		&= (97 \times 98 \times 99) \times 4 \\
		&= 941,094 \times 4 \\
		&=  \mathbf{3,764,376~ways}
		\end{split}
		\end{equation}
		
\pagebreak
		
		\item ผู้ที่ถือสลากหมายเลข 19, 47, 73 หรือ 97 เป็นหมายเลขของรางวัลทั้งสี่
		\begin{equation}
		\begin{split}
		_{96}C_{3} ~\times~ _{4}C_{1} &= \frac{96!}{(96 - 3)!} \times \frac{4!}{(4 - 1)!} \notag \\
		&= \frac{96!}{93!} \times \frac{4!}{3!} \\
		&= (94 \times 95 \times 96) \times 4 \\
		&= 857,280 \times 4 \\
		&=  \mathbf{3,429,120~ways}
		\end{split}
		\end{equation}
	\end{enumerate}
	
	% 6
	\item จงหาจำนวนเต็มบวกมีค่าไม่เกิน 1,000 ที่หารด้วย 7 หรือ 11 ลงตัว
	\begin{equation}
	\begin{split}
	A &= \{x ~|~ \frac{x}{7} = 0 ~\wedge~ x \leq 1,000 \} \notag \\
	|A| &= \lfloor \frac{1,000}{7} \rfloor \\
	&= 142 \\
	B &= \{y ~|~ \frac{y}{11} = 0 ~\wedge~ y \leq 1,000 \} \\
	|B| &= \lfloor \frac{1,000}{11} \rfloor \\
	&= 90 \\
	C &= \{z ~|~ z \in A ~\vee~ z \in B ~\wedge~ z \notin A \cap B\} \\
	|C| &= (|A| \cup |B|) - (|A| \cap |B|) \\
	&= |A| + |B| - |A \cap B| - |A \cap B| \\
	&= |A| + |B| - 2|A \cap B| \\
	&= 142 + 90 - 2(\lfloor \frac{1,000}{7 \times 11} \rfloor) \\
	&= 232 - 2(12) \\
	&= \mathbf{208~numbers}
	\end{split}
	\end{equation}
	
	% 7
	\item จงหาจำนวนวิธีเลือกไพ่ 5 ใบ โดยมีไพ่อย่างน้อย 1 ใบจากแต่ละ suit
	\begin{equation}
	\begin{split}
	(_{13}C_1 ~\times~ 3) ~+~ _{13}P_2 &= \left( \frac{13!}{(13 - 1)!} ~\times~ 3 \right) + \left( \frac{13!}{(13 - 2)! \cdot 2!} \right) \notag \\
	&= (13 \times 3) + \left( \frac{12 \cdot 13 }{2} \right) \\
	&= 39 \times \frac{156}{2} \\
	&= 39 \times 78 \\
	&= \mathbf{3,042~ways}
	\end{split}
	\end{equation}
	
\pagebreak
	
	% 8
	\item กำหนดให้ไพ่หนึ่งสำรับมี 52 ใบ จงหาจำนวนวิธีในการแจกไพ่จำนวน 5 ใบให้กับผู้เล่น 4 คน
	\begin{equation}
	\begin{split}
	_{52}P_5 ~+~ _{47}P_5 ~+~ _{42}P_5 ~+~ _{37}P_5 &= \frac{52!}{(52 - 5)! \cdot 5!} + \frac{47!}{(47 - 5)! \cdot 5!} + \frac{42!}{(42 - 5)! \cdot 5!} + \frac{37!}{(37 - 5)! \cdot 5!}\notag \\
	&= \frac{52!}{47! \cdot 5!} + \frac{47!}{42! \cdot 5!} + \frac{42!}{37! \cdot 5!} + \frac{37!}{32! \cdot 5!} \\
	&= \frac{48 \cdot 49 \cdot 50 \cdot 51 \cdot 52}{5!} + \frac{43 \cdot 44 \cdot 45 \cdot 46 \cdot 47}{5!} + \frac{38 \cdot 39 \cdot 40 \cdot 41 \cdot 42}{5!} \\
	&~~~+ \frac{33 \cdot 34 \cdot 35 \cdot 36 \cdot 37}{5!} \\
	&= \frac{311,875,200}{120} + \frac{184,072,680}{120} + \frac{102,080,160}{120} + \frac{52,307,640}{120} \\
	&= \frac{311,875,200 + 184,072,680 + 102,080,160 + 52,307,640}{120} \\
	&= \frac{650,335,680}{120} \\
	&= \mathbf{5,419,464~ways}
	\end{split}
	\end{equation}
	
	% 9
	\item จงหาจำนวนวิธีในการหยิบลูกบอลลักษณะเดียวกันจำนวน 10 ลูกใส่ในตะกร้า 8 ตะกร้าที่มีหมายเลข 1 ถึง 8\newline
	ติดอยู่ที่ด้านข้างตะกร้าแต่ละใบ
	\begin{equation}
	\begin{split}
	_{10}P_{8} &= \frac{10!}{(10 - 8)! \cdot 8!} \notag \\
	&= \frac{10!}{2! \cdot 8!} \\
	&= \frac{9 \times 10}{2} \\
	&= \mathbf{45~ways}
	\end{split}
	\end{equation}
	
	% 10
	\item จงหาจำนวนคำตอบที่เป็นไปได้ของสมการ $x_1 + x_2 + \dots + x_n = k$ เมื่อ $x_i \geq 0$
	\begin{equation}
	\begin{split}
	 \mathbf{\binom{k + n - 1}{n}~answers} \notag \\
	\end{split}
	\end{equation}
	
	% 11
	\item จาก Binomial theorem จงกระจายพจน์ทั้งหมดของ
	\begin{enumerate}
		\item ${(x + 1)}^4$
		\begin{equation}
		\begin{split}
		{(x + 1)}^4 &= \binom{4}{0} x^0 +  \binom{4}{1} x^1 + \binom{4}{2} x^2 + \binom{4}{3} x^3 + \binom{4}{4} x^4 \notag \\
		&=  \mathbf{1 + 4x + 6x^2 + 4x^3 + x^4}
		\end{split}
		\end{equation}
		
		\item ${(2 + y)}^4$
		\begin{equation}
		\begin{split}
		{(2 + y)}^4 &= \binom{4}{0} 2^4y^0 + \binom{4}{1} 2^3y^1 + \binom{4}{2} 2^2y^2 + \binom{4}{3} 2^1y^3 + \binom{4}{4} 2^0y^4 \notag \\
		&= 16 + (4 \times 8y) + (6 \times 4y^2) + (4 \times 2y^3) + y^4 \\
		&=  \mathbf{16 + 32y + 24y^2 + 8y^3 + y^4}
		\end{split}
		\end{equation}
	\end{enumerate}
	
\pagebreak
	
	% 12
	\item โยนเหรียญจำนวน 10 ครั้ง แต่ละครั้งออกหัวหรือก้อยด้วยความน่าจะเป็นที่เท่ากัน ให้หาค่าต่อไปนี้
	\begin{enumerate}
		\item จำนวน outcomes ที่เป็นไปได้ทั้งหมด
		\begin{equation}
		\begin{split}
		n(E) &= n(E_1) \times n(E_2) \times n(E_3) \times n(E_4) \times n(E_5) \times n(E_6) \times n(E_7) \times n(E_8) \times n(E_9) \times n(E_{10}) \notag \\
		&= 2^{10} \\
		&=  \mathbf{1,024~outcomes}
		\end{split}
		\end{equation}
		
		\item จำนวน outcomes ที่ออกหัวจำนวน 2 ครั้ง
		\begin{equation}
		\begin{split}
		n(E) &= (n(E_1) \times n(E_2) \times n(E_3) \times n(E_4) \times n(E_5) \times n(E_6) \times n(E_7) \times n(E_8)) \times 2 \notag \\
		&= 2^8 \times 2 \\
		&= 256 \times 2 \\
		&=  \mathbf{512~outcomes}
		\end{split}
		\end{equation}
		
		\item จำนวน outcomes ที่ออกก้อยอย่างมาก 3 ครั้ง
		\begin{equation}
		\begin{split}
		n(E) &= (n(E_1) \times n(E_2) \times n(E_3) \times n(E_4) \times n(E_5) \times n(E_6) \times n(E_7)) \times 3 \notag \\
		&= 2^7 \times 3 \\
		&= 128 \times 3 \\
		&=  \mathbf{384~outcomes}
		\end{split}
		\end{equation}
		
		\item จำนวน outcomes ที่ออกหัวและก้อยในจำนวนที่เท่ากัน
		\begin{equation}
		\begin{split}
		n(E) &= 5 \times 5 \notag \\
		&=  \mathbf{25~outcomes}
		\end{split}
		\end{equation}
	\end{enumerate}
\end{enumerate}

\end{document}