% !TEX program = xelatex

\documentclass[xetex,serif,aspectratio=1610]{beamer}

% \usepackage{newunicodechar}
% \usepackage{fontspec}
\usepackage{caption}
\usepackage{graphicx}
\usepackage{xunicode}
\usepackage{xltxtra}
\usepackage{pdfpages}
\usepackage{polyglossia}
\usepackage{hyperref}

\usetheme{Singapore}
% Marburg
% Pittsburgh
% Rochester
% Singapore
% Szeged

\usecolortheme{dove}
% crane
% dolphin
% dove
% seagull
% whale
% wolverine

\addtobeamertemplate{navigation symbols}{}{%
    \usebeamerfont{footline}%
    \usebeamercolor[fg]{footline}%
    \hspace{1em}%
    \href{https://github.com/naiithink}{@naiithink}~~\insertframenumber/\inserttotalframenumber
}

\AtBeginSection[]{%
  \begin{frame}
    \frametitle{Table of Contents}
    \tableofcontents[currentsection]
  \end{frame}
}

\AtBeginSubsection[]{%
  \begin{frame}
    \frametitle{Table of Contents}
    \tableofcontents[currentsection,currentsubsection]
  \end{frame}
}

\title[\LaTeX]{\Huge\textbf{\LaTeX{}}}
\subtitle{``A Document Preparation System''}
\author[naiithink]{พศวัต ถิ่นกาญจน์วัฒนา}
\date[CNC 2022]{17 สิงหาคม 2022}
\subject{CNC Update Tools presentation}

\XeTeXlinebreaklocale "th"
\setdefaultlanguage[numerals=arabic]{thai}
\setotherlanguages{english}

\setmainfont{TH Sarabun New}
\newfontfamily\thaifont[Script=Default]{TH Sarabun New}

\begin{document}
\maketitle
\appendix

\begin{frame}
\frametitle{\textbf{\LaTeX{}}~คืออะไร?}
``\textbf{\LaTeX{}}~is a high-quality typesetting system; it includes features designed
for the production of technical and scientific documentation. \textbf{\LaTeX{}}~is the
de facto standard for the communication and publication of scientific documents.
\textbf{\LaTeX{}}~is available as free software.''
\end{frame}

\begin{frame}
\frametitle{\textbf{\LaTeX{}}~ใช้ทำอะไร?}
\begin{columns}[c]
\column{.47\textwidth}
เตรียมเอกสาร เขียนหนังสือ
\column{.47\textwidth}
\begin{figure}
    \includegraphics[width=.55\textwidth]{./pdf/Laravel8-saacsos_cover.pdf}
    \caption*{หนังสือ Laravel 8 เขียนโดย @saacsos}
    \label{fig:book-saacsos-laravel8}
\end{figure}
\end{columns}
\end{frame}

\begin{frame}
\frametitle{\textbf{\LaTeX{}}~ใช้ทำอะไร?}
\begin{columns}[c]
\column{.47\textwidth}
ทำสไลด์นำเสนอ
\column{.47\textwidth}
\begin{figure}
    \includegraphics[width=.55\textwidth]{./pdf/Laravel8-saacsos_cover.pdf}
    \caption*{หนังสือ Laravel 8 เขียนโดย @saacsos}
    \label{fig:presentation-naiithink-slide-making}
\end{figure}
\end{columns}
\end{frame}

\begin{frame}
\frametitle{อยากใช้ \textbf{\LaTeX{}}~เริ่มยังไง?}
\begin{itemize}
    \item ติดตั้งบนคอมพิวเตอร์
    \item ใช้ผ่านเว็บไซต์ อย่าง \href{https://www.overleaf.com/}{overleaf.com} หรือ \href{https://replit.com}{replit.com}
    \item Docker Container \href{https://github.com/cnc2022/tex}{github.com/cnc2022/tex}
\end{itemize}
\end{frame}
\end{document}
