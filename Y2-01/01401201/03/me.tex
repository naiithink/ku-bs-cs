% !TEX program = xelatex

\documentclass{article}

% \usepackage{showframe}

\usepackage{blindtext}
\usepackage{fontspec}
\usepackage{geometry}
\usepackage{sidenotes}
\usepackage{wallpaper}
\usepackage{xunicode}
\usepackage{xltxtra}
\usepackage{xcolor}
\usepackage{multirow}
\usepackage{polyglossia}
\usepackage{hyperref}

\hypersetup{%
pdftitle = {ต้นนี้มันเป็นเรา},
pdfauthor = {พศวัต ถิ่นกาญจน์วัฒนา (6410451199)},
pdfkeywords = {01401201, พืชเพื่อการสร้างคุณค่าชีวิต}}

\reversemarginpar

\geometry{%
a4paper,
margin = 0.75in,
left = 1.28in,
right = 1.28in}

\XeTeXlinebreaklocale "th"
\setdefaultlanguage[numerals=arabic]{thai}
\setotherlanguages{english}

\setmainfont{TH Sarabun New}
\newfontfamily\thaifont[Script=Default]{TH Sarabun New}

\linespread{1.5}

\pagenumbering{gobble}

\title{\flushleft\Large พืชเพื่อการสร้างคุณค่าชีวิต\\
\Huge\textbf{ต้นนี้มันเป็นเรา}\\
\rule{0em}{1ex}\\
\normalsize พศวัต ถิ่นกาญจน์วัฒนา\\
รหัสนิสิต 6410451199
\vspace{-8ex}}
\author{}
\date{}

\begin{document}
\maketitle
\sloppy\flushleft
\textbf{กลุ่มพืชเมล็ดเปลือย เป็นพืชกลุ่มที่ผมรู้สึกถูกชะตาด้วยมากที่สุด} เพราะ พืชที่ผมชอบก็จัดอยู่ในพืชกลุ่มนี้

\rule{0em}{1ex}

ผมชอบต้นสน เพราะรู้สึกว่ามันสวยดี และเมื่อเราชอบเขาไปแล้ว เราก็อยากจะศึกษาและรู้จักเขาให้มากขึ้นกว่าเดิม (หมายถึงพืชนะครับ)
นอกจากนั้น ผมยังรู้สึกว่าพืชกลุ่มเมล็ดเปลือยนี้ยังมีความน่าสนใจตรงกลไกในการดูแลลูก\\
(ต้นอ่อน) ของตัวเอง ด้วยลักษณะของเมล็ดที่สามารถปกป้องให้ต้นอ่อนอยู่รอดได้ในสภาวะที่ไม่ค่อยจะเอื้ออำนวยต่อการอยู่รอดเท่าไรนัก
\textbf{มันแสดงออกถึงความใส่ใจในรายละเอียด} นอกจากความชอบในต้นสนแล้ว พืชกลุ่มนี้ก็เป็นพืชกลุ่มที่ผมเจออยู่ \mbox{บ่อย ๆ} อีกด้วย
อย่าง \textit{แปะก๊วย} เมล็ดสีเหลือง ๆ ผมเจอในขนมหวานอย่าง `เต้าทึง' (ผมชอบทานเต้าทึงมาก แต่ที่ผมจำแปะก๊วยได้ไม่ใช่ว่าผมชอบทานมันหรอกนะครับ
แต่เป็นเพราะว่าผมไม่ชอบทานมัน\mbox{ต่างหาก} ผมเขี่ยมันทิ้งทุกครั้งที่เจอเลย ผมเลยจำแปะก๊วยได้แม่นเลยครับว่าผมเกลียดมัน)
และผมก็เพิ่งได้รู้ว่าในประเทศไทยยังไม่สามารถปลูกแปะก๊วยเองได้จากในคลิปอาจารย์นี่แหละครับ ผมคิดว่าเราปลูกกันอย่างแพร่หลายมาก
เพราะที่บ้านผมมีหลาย ๆ เมนูที่ใช้แปะก๊วยเป็นวัตถุดิบ ผมเลยคิดว่ามันน่าจะหาได้ง่ายหรือปลูกเองได้ด้วยซ้ำ

\rule{0em}{1ex}

พอผมได้เรียนพืชกลุ่มนี้แล้ว ผมรู้สึกว่าได้เรียนสิ่งที่อยู่ใกล้ ๆ ตัว อะไรที่เคยคิดว่าน่าจะเป็นอีกแบบ ในความจริงกลับเป็นอีกแบบ (อย่างเรื่องการปลูกแปะก๊วยในประเทศไทย)
\textbf{และทั้งหมดนี้มันทำให้ผมรู้สึกว่าเรียนแล้วถูกชะตากับ\mbox{พืชกลุ่มนี้}มากที่สุดครับ}
\end{document}
