% !TEX program = xelatex

\documentclass{article}

% \usepackage{showframe}

\usepackage{blindtext}
\usepackage{fontspec}
\usepackage{geometry}
\usepackage{sidenotes}
\usepackage{xunicode}
\usepackage{xltxtra}
\usepackage{xcolor}
\usepackage{multirow}
\usepackage{polyglossia}
\usepackage{hyperref}

\hypersetup{%
pdftitle = {เราเจอพืชบ่อยแค่ไหน?},
pdfauthor = {พศวัต ถิ่นกาญจน์วัฒนา (6410451199)},
pdfkeywords = {01401201, พืชเพื่อการสร้างคุณค่าชีวิต}}

\reversemarginpar

\geometry{%
a4paper,
margin = 0.75in,
left = 1.6in}

\XeTeXlinebreaklocale "th"
\setdefaultlanguage[numerals=arabic]{thai}
\setotherlanguages{english}

\setmainfont{TH Sarabun New}
\newfontfamily\thaifont[Script=Default]{TH Sarabun New}

\linespread{1.5}

\pagenumbering{gobble}

% พศวัต ถิ่นกาญจน์วัฒนา

\title{\flushleft\Large พืชเพื่อการสร้างคุณค่าชีวิต\\
\Huge\textbf{เราเจอพืชบ่อยแค่ไหน?}\\
\rule{0em}{1ex}\\
\normalsize พศวัต ถิ่นกาญจน์วัฒนา\\
รหัสนิสิต 6410451199
\vspace{-8ex}}
\author{}
\date{}

\begin{document}
\maketitle
\sloppy\flushleft
วันวันหนึ่งผมเจอพืชบ่อยและเยอะจนจำไม่หมด อย่างวันนี้ผมเจอพืชแรกในจานอาหารเช้า ผมไปสืบมาได้ความว่ามันคือ ``กรีนโอ๊ค''\sidenote[]{กรีนโอ๊ค}
บางทีผมก็ชอบเอามันมาทานเล่น เคี้ยวแล้วกรอบอร่อยดี พืชถัดไปผมเจอตอนเดินไปคณะวิทยาศาสตร์อันเป็นที่รัก
มันคือหญ้า\sidenote[]{หญ้า} ที่ขึ้นอยู่บนทางเท้า และผมเห็นมันเพราะผมต้องมองทางเวลาเดิน ไม่อย่างนั้น ผมจะเดินชนพืชถัดไปที่ผมได้เจอ
นั่นคือต้นปาล์ม\sidenote[]{ต้นปาล์ม} ที่เรียงกันอยู่จำนวน 5 ต้นที่หน้าธนาคารแถวประตูงามวงศ์วาน 1

\rule{0em}{1ex}

ผมเดินต่อไป เดินเข้าคณะวิทยาศาสตร์ รอบนี้ผมไม่ได้เจอแค่พืชนะครับ แต่ผมเจอตึกภาควิชาพฤกษศาสตร์ ซึ่งมีพืชปลูกไว้ขนาบสองข้างบันไดทางขึ้นตึก
แต่ผมก็ได้เดินผ่านร้านกาแฟมาด้วย กาแฟ\sidenote[]{กาแฟ} ก็เป็นพืชชนิดหนึ่งเหมือนกัน ข้างในร้านกาแฟก็มีหลายเมนูที่มีพืชน่าสนใจ อย่างช็อกโกแลตก็ทำจากต้นโกโก้\sidenote[]{โกโก้}
ชาเขียวจากต้นชา\sidenote[]{ชา} น้ำตาลจากอ้อย\sidenote[]{อ้อย} ขนมปังจากข้าว\sidenote[]{ข้าว} ซอสมะเขือเทศก็คงต้องทำจากมะเขือเทศ\sidenote[]{มะเขือเทศ}
และซอสพริกก็คงจะทำจากมะเขือเทศไม่ได้ ต้องทำจากพริก\sidenote[]{พริก} กระดาษที่ใช้ทำเป็นภาชนะก็ทำจากยูคาลิปตัส\sidenote[]{ยูคาลิปตัส}
แล้วจู่ ๆ ผมก็อยากดื่มน้ำส้ม\sidenote[]{ส้ม} เอ๊ะ ผมเพิ่งนึกได้ว่าในจานอาหารเช้าที่ได้ทานไปมีแคร์รอท\sidenote[]{แคร์รอท} สีส้มอยู่ด้วยนะ

\rule{0em}{1ex}

เย็นวันเดียวกัน ผมเดินกลับห้องหลังเลิกเรียนวิชาสุดท้าย ฝนฟ้าไม่ค่อยเป็นใจตกกระหน่ำ พอถึงห้องแล้วก็พบว่าชุดนิสิตที่ทำจากฝ้าย\sidenote[]{ฝ้าย} เปียกโชกเลย
ผมรีบจัดการตัวเองแล้วสั่งอะไรมาทานด้วยความหิว แล้วหลังจากที่รออยู่นาน อาหารก็มาถึง เปิดกล่องมาเจองา\sidenote[]{งา} ที่โรยอยู่บนแฮมเบอร์เกอร์
หันซ้ายไปก็เจอมันฝรั่ง\sidenote[]{มันฝรั่ง} ทอดสีเหลืองทอง ผมทานจนอิ่มแล้วเดินเอาขยะไปทิ้ง อ้าว ที่ถังขยะเจอกล่องอาหารของใครไม่รู้ที่ข้างในมีผักชี\sidenote[]{ผักชี} เหลืออยู่ก้นกล่อง เจอพืชอีกแล้ว
\end{document}
