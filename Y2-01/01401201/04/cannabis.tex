% !TEX program = xelatex

\documentclass{article}

% \usepackage{showframe}

% \usepackage{blindtext}
\usepackage{fontspec}
\usepackage{geometry}
\usepackage{sidenotes}
% \usepackage{wallpaper}
\usepackage{xunicode}
\usepackage{xltxtra}
\usepackage{xcolor}
\usepackage{multirow}
\usepackage{polyglossia}
\usepackage{hyperref}

\hypersetup{%
pdftitle = {กัญชา กัญไหม},
pdfauthor = {พศวัต ถิ่นกาญจน์วัฒนา (6410451199)},
pdfkeywords = {01401201, พืชเพื่อการสร้างคุณค่าชีวิต}}

\reversemarginpar

\geometry{%
a4paper,
margin = 0.75in,
left = 1.28in,
right = 1.28in}

\XeTeXlinebreaklocale "th"
\setdefaultlanguage[numerals=arabic]{thai}
\setotherlanguages{english}

\setmainfont{TH Sarabun New}
\newfontfamily\thaifont[Script=Default]{TH Sarabun New}

\linespread{1.5}

\pagenumbering{gobble}

\title{\flushleft\Large พืชเพื่อการสร้างคุณค่าชีวิต\\
\Huge\textbf{กัญชา กัญไหม}\\
\rule{0em}{1ex}\\
\normalsize พศวัต ถิ่นกาญจน์วัฒนา\\
รหัสนิสิต 6410451199
\vspace{-8ex}}
\author{}
\date{}

\begin{document}
\maketitle
\sloppy\flushleft
หากอ้างอิงจากคลิปที่ 4 (\href{https://youtu.be/ARtuiHrqm5Y}{กัญชง Vs กัญชา - Money Chat Thailand!})\\
พืชที่เป็นประเด็นในการปลดออกจากบัญชีสารเสพติด คือ \textbf{\textit{กัญชง (Hemp)}} ซึ่งมี CBD อยู่สูง\\
\underline{ไม่ใช่} \textbf{\textit{กัญชา (Marijuana)}} ที่มี THC อยู่สูง

\rule{0em}{1ex}

\begin{tabular}{c l}
\textbf{THC}    & ก่อให้เกิดความรู้สึกเคลิบเคลิ้ม หรือที่เรียกว่า \textit{เมากัญชา} ในทางการพาณิชย์มีมูลค่าสูง\\
\textbf{CBD}    & สามารถ \underline{ช่วย} ลดการเจ็บปวด ผ่อนคลายประสาท สามารถสกัดเป็นยาช่วยลดอาการลมชักในคนได้
\end{tabular}

\rule{0em}{1ex}

\textbf{ในความคิดเห็นของผม} การปลด Cannabis ออกจากบัญชีสารเสพติดเพื่อการรักษาพยาบาล ไม่ใช่เพื่อการบันเทิงเป็นเรื่องที่ดี
หากการเก็บผลผลิตเป็นไปตามเกณฑ์ที่กำหนดไว้ กล่าวคือ หากวัด Cannabinoids แล้ว ในการสกัด CBD จากกัญชง
จะต้องมี THC $\le$ 0.2~\% ซึ่งหากเป็นไปตามนี้ จะไม่ถือว่าเป็นสารเสพติด

\rule{0em}{1ex}

ส่วนการปลด Cannabis ออกจากบัญชีสารเสพติดเพื่อประโยชน์ในด้านอื่นนั้น ผมไม่สามารถออกความคิดเห็นได้
เนื่องจากยังไม่ได้ศึกษาข้อมูลมากเพียงพอจนรู้สึกว่าสามารถออกความคิดเห็นอย่างตรงไปตรงมาได้ แต่ผมมีเงื่อนไข\\
อยู่ว่า \textbf{ถ้าหากมันก่อให้เกิดความเดือดร้อนแก่ใครก็ตามแม้นแต่คนเดียว ไม่ว่าจะในทางตรงหรือทางอ้อม}\\
การใช้ประโยชน์ในด้านนั้นจำเป็นจะต้องมีการควบคุมไม่ให้เกิดความเดือดร้อนขึ้นครับ

\rule{0em}{6ex}

\fbox{\begin{minipage}[c][2.4em]{\textwidth}
\centering
สัปดาห์นี้ผมขอตอบแบบซีเรียสมากกว่าสัปดาห์ก่อน ๆ\\
เพราะผมคิดว่าเป็นเรื่องที่มีผลกระทบค่อนข้างมากทั้งในด้านบวกและด้านลบครับ
\end{minipage}}
\end{document}
