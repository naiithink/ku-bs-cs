% !TEX program = xelatex

\documentclass{article}

% \usepackage{showframe}

% \usepackage{blindtext}
\usepackage{fontspec}
\usepackage{geometry}
\usepackage{sidenotes}
% \usepackage{wallpaper}
\usepackage{xunicode}
\usepackage{xltxtra}
\usepackage{xcolor}
\usepackage{multirow}
\usepackage{polyglossia}
\usepackage{hyperref}

\hypersetup{%
pdftitle = {วิชาปรุงยา},
pdfauthor = {พศวัต ถิ่นกาญจน์วัฒนา (6410451199)},
pdfkeywords = {01401201, พืชเพื่อการสร้างคุณค่าชีวิต}}

\reversemarginpar

\geometry{%
a4paper,
margin = 0.75in,
left = 1.28in,
right = 1.28in}

\XeTeXlinebreaklocale "th"
\setdefaultlanguage[numerals=arabic]{thai}
\setotherlanguages{english}

\setmainfont{TH Sarabun New}
\newfontfamily\thaifont[Script=Default]{TH Sarabun New}

\linespread{1.5}

\pagenumbering{gobble}

\title{\flushleft\Large พืชเพื่อการสร้างคุณค่าชีวิต\\
\Huge\textbf{วิชาปรุงยา}\\
\rule{0em}{1ex}\\
\normalsize พศวัต ถิ่นกาญจน์วัฒนา\\
รหัสนิสิต 6410451199
\vspace{-8ex}}
\author{}
\date{}

\begin{document}
\maketitle
\sloppy\flushleft

สมุนไพรที่ผมเคยใช้คือ \textbf{มะขามป้อม}, \textbf{\textit{Phyllanthus emblica}}, โดยผม\textbf{ใช้รับประทานเพื่อบรรเทาอาการระคายคอ} และที่ผมเลือกใช้สมุนไพรชนิดนี้\textbf{เพราะคุณแม่ให้ผมทานครับ}

\rule{0em}{1ex}

เมื่อตอนที่ผมเรียนอยู่ประถมปลาย มีอยู่วันหนึ่งผมระคายคอและไอตั้งแต่เช้า สาย บ่าย เย็น จนกระทั่งมืด คุณแม่ของผมคงเห็นอาการของผม (รวมทั้งเพื่อนร่วมห้องของผมกว่า 40 คนนั้นด้วย)
เย็นวันถัดมา คุณแม่ของผมได้ยื่นขวดพลาสติกใส ฝาสีเหลือง ข้างในบรรจุเม็ดกลม ๆ สีน้ำตาลเข้มนับร้อยเม็ด แต่ละเม็ดขนาดเป็นสองเท่าของหัวปากกาลูกลื่นขนาดครึ่งมิลลิเมตร
ตอนนั้นที่ผมยังเป็นเด็ก สิ่งที่อยู่ในขวดมันดูน่ากลัวมากนะครับ ผมได้ลองทานไปเป็นจำนวนหนึ่งเม็ดถ้วน รสชาติมันจะเปรี้ยวก็ไม่เชิง หวานนิด ๆ ขมปลาย ๆ ออกมาก็ไม่ได้อูมามิ แต่ก็ไม่ได้เลวร้ายอะไร
ส่วนกลิ่น สำหรับผมไม่หอมแต่ก็ไม่เหม็นครับ

\rule{0em}{1ex}

ผมรู้สึกได้ว่าอาการระคายคอของผมดีขึ้นในช่วงเวลาที่เม็ดมะขามป้อมที่อมอยู่นั้นยังไม่ละลายจนหมด แต่พอมันละลายจนหมด อาการก็เริ่มกลับมา ผมเลยเข้าใจว่ามะขามป้อมมีความสามารถในการบรรเทาอาการ
ไม่ใช่รักษาอาการ ไม่กี่วันหลังจากนั้น อาการระคายคอของผมก็ค่อย ๆ ดีขึ้นครับ
\end{document}
