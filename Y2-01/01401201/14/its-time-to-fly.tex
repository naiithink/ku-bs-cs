% !TEX program = xelatex

\documentclass{article}

% \usepackage{showframe}

% \usepackage{blindtext}
\usepackage{fontspec}
\usepackage{geometry}
\usepackage[absolute,overlay]{textpos}
\usepackage{sidenotes}
\usepackage{wallpaper}
\usepackage{xunicode}
\usepackage{xltxtra}
\usepackage{xcolor}
\usepackage{multirow}
\usepackage[bottom,hang]{footmisc}
\usepackage{polyglossia}
\usepackage{hyperref}

\hypersetup{%
pdftitle = {ไปเที่ยวกันไหม},
pdfauthor = {พศวัต ถิ่นกาญจน์วัฒนา (6410451199)},
pdfkeywords = {01401201, พืชเพื่อการสร้างคุณค่าชีวิต}}

\reversemarginpar

\geometry{%
a4paper,
margin = 0.75in,
left = 1.28in,
right = 1.28in}

\XeTeXlinebreaklocale "th"
\setdefaultlanguage[numerals=arabic]{thai}
\setotherlanguages{english}

\setmainfont{TH Sarabun New}
\newfontfamily\thaifont[Script=Default]{TH Sarabun New}

\linespread{1.5}

\pagenumbering{gobble}

\definecolor{green}{RGB}{93, 201, 73}

\title{\flushleft\Large พืชเพื่อการสร้างคุณค่าชีวิต\\
\Huge\textbf{ไปเที่ยวกันไหม}\\
\rule{0em}{1ex}\\
\normalsize พศวัต ถิ่นกาญจน์วัฒนา\\
รหัสนิสิต 6410451199
\vspace{-8ex}}
\author{}
\date{}

\addtolength{\wpYoffset}{-9cm}

\begin{document}
\CenterWallPaper{1}{images/48954139137_58c801ccbd_k.jpg}
\begin{textblock*}{9cm}(10.6cm, 28cm)
\color{white} Artist's rendering of Starship taking off on Mars~ | ~\textbf{SpaceX}
\end{textblock*}
\maketitle
\sloppy\flushleft

\textbf{ผมอยากลองไปเที่ยวดูสถานที่ปลูกพืชที่คนส่วนใหญ่ไม่เคยไปดูครับ}

\rule{0em}{0.8ex}

ผมได้ดูคลิปคลิปหนึ่ง ในคลิปก็ไม่มีอะไรนอกจากคน ๆ หนึ่งกำลังเด็ดผักกาดหอมด้วยท่าทีที่สุขุม แล้วอยู่ ๆ ก็มีคนเพิ่มมาอีกสองคน
รวมกันตั้งวงกินผักกาดหอม แล้วก็จบคลิป ฟังดูแล้วก็แค่คลิปคนสามคนตั้งปาร์ตีกินผักกาดหอมธรรมดา อย่างงง ๆ ดูไม่มีอะไร
แต่ที่มันไม่ปกติก็คือ คลิปนี้ไม่ได้ถูกถ่ายบนพื้นโลกนี่สิครับ คลิปที่ผมกำลังพูดถึงมีชื่อว่า \href{https://youtu.be/RqtAK-FBtXU}{``Space in 4K - First Lettuce Grown and Eaten in Space''}
ถูกอัปโหลดตั้งแต่ปี 2015 หรือเมื่อ 7 ปีก่อน โดย \mbox{\href{https://www.youtube.com/user/ReelNASA}{NASA Johnson}}
สิ่งที่ผมสนใจในคลิปนี้ แน่นอนว่าคือการปลูกและเก็บเกี่ยวพืชในอวกาศเพื่อนำมารับประทาน แล้วถ้าลองนึกดู ตอนนี้เราสามารถปลูกพืชในอวกาศได้แล้ว
แต่ดูแล้วก็ได้แค่พืชต้นเล็ก ๆ แบบนี้ต่อไปเราจะปลูกพืชที่ต้นใหญ่กว่านี้ได้ไหม ถ้าไม่ได้ อะไรคือสาเหตุ

\rule{0em}{0.8ex}

มากไปกว่านั้น ผมกำลังสนใจ \href{https://www.spacex.com/human-spaceflight/mars/}{The Mars Mission ของ SpaceX} ที่จะไปตั้งอารยธรรมของมนุษย์บนดาวอังคาร
และเมื่อไปตั้งถิ่นฐานที่นั่น ก็ต้องปลูกพืชเพื่อใช้ประโยชน์ต่าง ๆ ถ้าเลือกได้ \textbf{ผมเลยอยากไปเที่ยวดูพืชที่นอกโลกดูครับ}
\end{document}
