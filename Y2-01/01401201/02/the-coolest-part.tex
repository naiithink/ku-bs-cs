% !TEX program = xelatex

\documentclass{article}

% \usepackage{showframe}

\usepackage{blindtext}
\usepackage{fontspec}
\usepackage{geometry}
\usepackage{sidenotes}
\usepackage{wallpaper}
\usepackage{xunicode}
\usepackage{xltxtra}
\usepackage{xcolor}
\usepackage{multirow}
\usepackage{polyglossia}
\usepackage{hyperref}

\hypersetup{%
pdftitle = {ส่วนไหนดี?},
pdfauthor = {พศวัต ถิ่นกาญจน์วัฒนา (6410451199)},
pdfkeywords = {01401201, พืชเพื่อการสร้างคุณค่าชีวิต}}

\reversemarginpar

\geometry{%
a4paper,
margin = 0.75in,
left = 1.28in,
right = 1.28in}

\XeTeXlinebreaklocale "th"
\setdefaultlanguage[numerals=arabic]{thai}
\setotherlanguages{english}

\setmainfont{TH Sarabun New}
\newfontfamily\thaifont[Script=Default]{TH Sarabun New}

\linespread{1.5}

\pagenumbering{gobble}

\title{\flushleft\Large พืชเพื่อการสร้างคุณค่าชีวิต\\
\Huge\textbf{ส่วนไหนดี?}\\
\rule{0em}{1ex}\\
\normalsize พศวัต ถิ่นกาญจน์วัฒนา\\
รหัสนิสิต 6410451199
\vspace{-8ex}}
\author{}
\date{}

\addtolength{\wpYoffset}{-7.9cm}

\begin{document}
\CenterWallPaper{1}{images/pexels-quang-nguyen-vinh-2149105.jpg}
\maketitle
\sloppy\flushleft
\textbf{สำหรับผม ส่วนที่ \textit{เท่} ที่สุดของพืชคือ `ใบ' เพราะมันคือส่วนที่เมื่อเทียบกับคนแล้ว ผมมองว่ามันคล้ายกับ `ผม'} (\mbox{หมายถึง} \textit{hair} n. นะครับ ไม่ใช่ตัวผม)
แล้วถ้าลองนึกดู คนเราสามารถเสริม \textit{ความเท่} ด้วยการตัดแต่งทรงผมได้ นอกจากเรื่องของความสวยงามแล้ว เรื่องของการใช้งาน (function) ก็ \textit{เท่} ไม่แพ้กัน ด้วยความสามารถของใบ
ที่หลากหลาย ทั้งสามารถควบคุมปริมาณของสารต่าง ๆ ให้อยู่ระดับที่พอดีได้ อีกทั้งยังสามารถสร้างพลังงานให้กับตัวเองได้โดยใช้แสง น้ำ และคาร์บอนไดออกไซด์
เรียกได้ว่า \textbf{แค่ยืนคูล ๆ \textit{เท่ ๆ} แทบไม่ต้องขยับตัวก็ยังอิ่มได้} นอกจากที่พืชจะใช้ประโยชน์จากใบของตัวเองแล้ว
คนเราก็เอาใบของพืชมาใช้ประโยชน์ได้อีก อย่าง เอามาห่อขนม เอามาทำเครื่องจักสาน (ใบหวาย กก ก้านมะพร้าว ตาล ใบลาน ย่านลิเภา ลำเจียก)
\end{document}
