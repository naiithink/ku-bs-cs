% !TEX program = xelatex

\documentclass{article}

% \usepackage{showframe}

% \usepackage{blindtext}
\usepackage{fontspec}
\usepackage{geometry}
\usepackage{sidenotes}
% \usepackage{wallpaper}
\usepackage{xunicode}
\usepackage{xltxtra}
\usepackage{xcolor}
\usepackage{multirow}
\usepackage[bottom,hang]{footmisc}
\usepackage{polyglossia}
\usepackage{hyperref}

\hypersetup{%
pdftitle = {คนมา ป่าต้องไม่หมด},
pdfauthor = {พศวัต ถิ่นกาญจน์วัฒนา (6410451199)},
pdfkeywords = {01401201, พืชเพื่อการสร้างคุณค่าชีวิต}}

\reversemarginpar

\geometry{%
a4paper,
margin = 0.75in,
left = 1.28in,
right = 1.28in}

\XeTeXlinebreaklocale "th"
\setdefaultlanguage[numerals=arabic]{thai}
\setotherlanguages{english}

\setmainfont{TH Sarabun New}
\newfontfamily\thaifont[Script=Default]{TH Sarabun New}

\linespread{1.5}

\pagenumbering{gobble}

\definecolor{green}{RGB}{93, 201, 73}

\title{\flushleft\Large พืชเพื่อการสร้างคุณค่าชีวิต\\
\Huge\textbf{คนมา {\color{green}ป่าต้องไม่หมด}}\\
\rule{0em}{1ex}\\
\normalsize พศวัต ถิ่นกาญจน์วัฒนา\\
รหัสนิสิต 6410451199
\vspace{-8ex}}
\author{}
\date{}

\begin{document}
\maketitle
\sloppy\flushleft

เนื่องจากป่าไม่ใช่สิ่งที่ต้องใช้เวลาสร้างนานเท่าเชื้อเพลิงจากซากดึกดำบรรพ์\\
\textbf{ผมจึงเสนอให้ปลูกต้นไม้เองเพื่อตัดมาใช้งาน (ทำฟาร์มต้นไม้) หรือ จำกัดปริมาณการตัดต้นไม้ ปลูกไม้กลับคืนในปริมาณที่เหมาะสม และ ณ เวลาที่เหมาะสมครับ}

\rule{0em}{1ex}

เมื่อก่อนผมเรียนมาตลอดว่าการตัดไม้ทำลายป่าเป็นสิ่งไม่ดี ตอนนี้ผมคิดว่าประโยคนี้ไม่ได้ถูกต้องเสมอไป
เพราะในความคิด หากคิดแบบอุดมคติ การตัดไม่ทำลายป่าจะไม่ได้สร้างความเสียหายมากมายถ้ามีการจำกัดปริมาณในการตัด มีการปลูกคืนในระยะเวลาที่เหมาะสม
และในปริมาณที่เหมาะสม หากลองคิดดู ถ้าเราบอกว่าแค่ว่า ``ปลูกคืนในปริมาณที่เหมาะสม'' เราคิดแค่เรื่องของปริมาณ แต่ลืมไปว่าต้นไม้เป็นสิ่งมีชีวิต
ที่ต้องใช้เวลาในการเติบโต ถ้าเราตัดต้นไม้หมดป่าในวันนี้ แล้วพรุ่งนี้มาปลูกใหม่ กว่าต้นไม้ที่ปลูกไปจะโตมาช่วยเติมเต็มวัฏจักรที่สำคัญต่อสิ่งมีชีวิต
สิ่งมีชีวิตบางอย่างก็อาจถูกทำลายหรือเสียหายได้ แต่ชีวิตจริงอาจไม่ใช่ชีวิตในอุดมคติ การดูแลควรคุมการตัดต้นไม้เป็นอะไรที่ยากที่จะทำ
รวมถึงการคำนวณอัตราการปลูกคืนและระยะเวลาอาจไม่ใช่เรื่องที่ทุกคนทำได้ หรือมีเวลาเพียงพอที่จำใส่ใจ หรือแม้นแต่คิดที่จะสนใจ
เมื่อชีวิตจริงเป็นแบบนี้ หากคิดแบบง่ายที่สุด ทางออกที่ดีที่สุดก็คงจะเป็น ห้ามไม่ให้ตัดไม้ทำลายป่าตั้งแต่แรก

\rule{0em}{1ex}

แนวคิดในการแก้ปัญหาเกิดขึ้นมาใหม่ได้ตลอดเวลา คนอื่นอาจจะมีแนวคิดที่สร้างสรรค์กว่าความคิดของผม อย่าง \href{https://www.ecosia.org/}{\textbf{Ecosia}} ที่มีจุดหมายว่า
\textbf{จะปลูกต้นไม้ให้กับที่ที่ต้องการ จากผลกำไรที่ได้จากการใช้งานของเรา -- ``We use the profit we make from your searches to plant trees where they are needed most.''}\footnotemark[1]

\footnotetext[1]{\textit{What is Ecosia? - The search engine that plants trees}. Ecosia. Retrieved October 2, 2022,\\
\mbox{from \href{https://info.ecosia.org/what/}{https://info.ecosia.org/what/}}}
\end{document}
