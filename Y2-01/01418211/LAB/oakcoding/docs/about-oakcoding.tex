% !TEX program = xelatex

\documentclass{article}

\usepackage{fontspec}
\usepackage{geometry}
\usepackage{xunicode}
\usepackage{xltxtra}
\usepackage{multirow}
\usepackage{polyglossia}
\usepackage{fancyhdr}
\usepackage{hyperref}

\newcommand{\thegroup}{OakCoding}

\hypersetup{%
pdftitle = {ข้อมูลกลุ่มโครงงาน Software Construction \textendash \thegroup},
pdfsubject = {ข้อมูลเกี่ยวกับกลุ่ม \thegroup~โครงงานวิชา 01418211 Software Construction ปีการศึกษา 2565 ภาคต้น},
pdfauthor = {{6410450176 ปาณชัย}, {6410451041 ธนากร}, {6410451067 ธเนศ}, {6410451199 พศวัต}},
pdfkeywords = {\thegroup, 01418211, Software Construction}}

\geometry{%
a4paper,
margin = 0.75in}

\XeTeXlinebreaklocale "th"
\setdefaultlanguage[numerals=arabic]{thai}
\setotherlanguages{english}

\setmainfont{TH Sarabun New}
\newfontfamily\thaifont[Script=Default]{TH Sarabun New}

% \linespread{1.5}

\setlength\headheight{0.75in}

\pagestyle{fancy}
\fancyhead{}
\fancyhead[R]{\textbf{\thegroup}~\thepage}
\pagenumbering{gobble}

\title{\flushleft\vspace{-5ex}\Huge\textbf{ข้อมูลกลุ่มโครงงาน Software Construction}\\
\Large 01418211 Software Construction ปีการศึกษา 2565 ภาคต้น
\vspace{-8ex}}
\author{}
\date{}

%%%%% BEGIN DOCUMENT %%%%%

\begin{document}
\maketitle
\thispagestyle{fancy}
\sloppy\flushleft

สร้าง Desktop Application สําหรับการแจ้งเรื่องร้องเรียนของนิสิตมหาวิทยาลัยเกษตรศาสตร์ ด้วย JavaFX (JavaSE 17 เท่านั้น)\\
โดยต้องออกแบบและเขียนโปรแกรมที่ใช้หลักการโปรแกรมเชิงวัตถุ

\section{เกี่ยวกับกลุ่ม}
\begin{tabular}{l l l}
\textbf{ชื่อกลุ่ม}                      & \thegroup                 &\\
\textbf{จำนวนสมาชิก}                 & 4                         &\\
\textbf{ตัวแทนกลุ่ม}                   & พศวัต ถิ่นกาญจน์วัฒนา          &\\
\textbf{Repository URI}             & \href{https://github.com/CS211-651/project211-oakcoding.git}{https://github.com/CS211-651/project211-oakcoding.git}   & (GitHub Classroom)\\
                                    & \href{https://github.com/naiithink/oakcoding.git}{https://github.com/naiithink/oakcoding.git}                         & (Development)
\end{tabular}

\subsection{รายชื่อสมาชิก}
\begin{tabular}{c l l l l l}
\hline
\textbf{ลำดับ}    & \textbf{ชื่อ​}     & \textbf{นามสกุล}           & \textbf{GitHub Username}                                           & \textbf{รหัสนิสิต}       & \textbf{KU E-mail Address}\\
\hline
1               & ปาณชัย            & คชกาษร                    & \href{https://github.com/ingfosbreak}{@ingfosbreak}                & 6410450176            & panachai.ko@ku.th\\
                & Panachai         & Kotchagason               &                                                                    &                       &\\
2               & ธนากร            & คนหมั่น                     & \href{https://github.com/Thanakorn0Khonman}{@Thanakorn0Khonman}    & 6410451041            & thanakorn.khon@ku.th\\
                & Thanakorn        & Khonman                   &                                                                    &                       &\\
3               & ธเนศ             & จีนสีคง                     & \href{https://github.com/thanetjin}{@thanetjin}                    & 6410451067            & thanet.jin@ku.th\\
                & Thanet           & Jinseekhong               &                                                                    &                       &\\
4               & พศวัต             & ถิ่นกาญจน์วัฒนา               & \href{https://github.com/naiithink}{@naiithink}                    & 6410451199            & potsawat.t@ku.th\\
                & Potsawat         & Thinkanwatthana           &                                                                    &                       &\\
\hline
\end{tabular}

\section{รับทราบเงื่อนไขการได้เกรด F}
\begin{enumerate}
    \setlength\itemsep{0ex}
    \item โครงงานที่สมบูรณ์ของนิสิตกลุ่มใดที่ไม่ผ่านเกณฑ์ที่กำหนด สมาชิกทุกคนในกลุ่มจะได้เกรด F
    \item นิสิตกลุ่มใดที่ไม่นำเสนอความก้าวหน้าครั้งใดก็ตาม ภายในเวลาที่กำหนด สมาชิกทุกคนในกลุ่มจะได้เกรด F
    \item การนำเสนอความก้าวหน้าครั้งใดก็ตามหรือการส่งโครงงานที่สมบูรณ์ นิสิตคนใดที่ไม่มี commit หรือ commit นั้นไม่มีการเขียนโปรแกรมอย่างมีนัยสำคัญ\\
          นิสิตคนนั้นจะได้เกรด F โดยไม่มีการเปลี่ยนแปลงรายชื่อสมาชิกของกลุ่ม
\end{enumerate}
\end{document}
