% !TEX program = xelatex

\documentclass{article}

% \usepackage{showframe}

% \usepackage{blindtext}
\usepackage{fontspec}
\usepackage{geometry}
\usepackage{sidenotes}
% \usepackage{wallpaper}
\usepackage{xunicode}
\usepackage{xltxtra}
\usepackage{amsmath}
\usepackage{xcolor}
\usepackage{multirow}
\usepackage{listings}
\usepackage{polyglossia}
\usepackage{hyperref}

\hypersetup{%
pdftitle = {Recursion: Factorial},
pdfauthor = {พศวัต ถิ่นกาญจน์วัฒนา (6410451199)},
pdfkeywords = {01418231, {data structures}, recursion}}

\reversemarginpar

\geometry{%
a4paper,
margin = 0.55in,
left = 1in,
right = 1in}

\XeTeXlinebreaklocale "th"
\setdefaultlanguage[numerals=arabic]{thai}
\setotherlanguages{english}

\setmainfont{TH Sarabun New}
\newfontfamily\thaifont[Script=Default]{TH Sarabun New}

\lstset{%
basicstyle = \ttfamilylatin,
commentstyle = \color{olive},
keywordstyle = \color{magenta},
numbers = left,
captionpos = b,
belowcaptionskip = 1ex,
frame = single,
xleftmargin = 3.4pt,
xrightmargin = -3.4pt}

\linespread{1.5}

\pagenumbering{gobble}

\title{\flushleft\normalsize 01418231, Data Structures\\
\LARGE \textbf{Recursion}: Factorial\\
\vspace{2ex}
\normalsize พศวัต ถิ่นกาญจน์วัฒนา\\
รหัสประจำตัวนิสิต 6410451199
\vspace{-12ex}}
\author{}
\date{}

\begin{document}
\maketitle
\sloppy\flushleft

\fbox{\begin{minipage}[c]{\textwidth}
วาดแผนภาพ แสดงการเรียกซ้ำ ของ function factorial แบบ recursive
\end{minipage}}

\begin{minipage}[t]{\textwidth}
\begin{lstlisting}[language=C,caption=The Factorial Function]
int factorial(int n)
{
    if (n == 1)
        return 1;
    else
        return n * factorial(n - 1);
}
\end{lstlisting}
\end{minipage}

\rule{0em}{5ex}

\begin{lstlisting}[language=C,numbers=none,frame=none]
int result = 0;
result = factorial(5);

factorial(5)    =    5  *  factorial(4)
                           \__________/
                            |
                            |
                =    5  *   ( 4  *  factorial(3) )
                                    \__________/
                                     |
                                     |
                =    5  *     4  *   ( 3  *  factorial(2) )
                                             \__________/
                                              |
                                              |
                =    5  *     4  *     3  *   ( 2  *  factorial(1) )
                                                      \__________/
                                                       |
                                                       |
                =    5  *     4  *     3  *     2  *   ( 1 )
                     \_____________________________________/
                      |
                      |
                =     120

result = 120;
\end{lstlisting}

\end{document}
